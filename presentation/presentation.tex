\documentclass[10pt, compress]{beamer}

\usetheme[usetitleprogressbar]{utopia}

\usepackage{configfrancais}
\usepackage[french]{babel}
\usepackage{booktabs}
\usepackage[scale=2]{ccicons}
\usepackage{minted}

\usepgfplotslibrary{dateplot}

\usemintedstyle{trac}

\title{Combinatoire et chiffrement : le père Noël secret et ses applications}
\subtitle{}
\date{2021|22}
\author{Stanislas MEZUREUX | 33920}
\institute{Lycée Henri POINCARÉ}

\begin{document}

\maketitle

\begin{frame}
  \frametitle{Sommaire}
  \tableofcontents
\end{frame}

\section{Position du problème}

  \begin{frame}{Père Noël secret ?}
    \begin{minipage}[h]{0.48\textwidth}
      \begin{figure}
        \begin{tikzpicture}[main/.style = {draw, circle}, scale=1.5, transform shape] 
          \node[main] (1) {$p_1$};
          \node[main] (2) [right of=1] {$p_2$};
          \node[main] (3) [below of=2] {$p_3$};
          \node[main] (4) [left of=3] {$p_4$};
          \draw[-latex] (1) to [out=90, in=90, looseness=1] (2);
          \draw[-latex] (2) to [out=0, in=0, looseness=1] (3);
          \draw[-latex] (3) to [out=270, in=270, looseness=1] (4);
          \draw[-latex] (4) to [out=180, in=180, looseness=1] (1);
        \end{tikzpicture} 
        \caption{Un groupe}
      \end{figure}
    \end{minipage}
    \begin{minipage}[h]{0.48\textwidth}
      \begin{figure}
        \begin{tikzpicture}[main/.style = {draw, circle}, scale=1.5, transform shape] 
          \node[main, fill=methodBox!50] (1) {$p_1$};
          \node[main, fill=methodBox!50] (2) [right of=1] {$p_2$};
          \node[main] (3) [below of=2] {$p_3$};
          \node[main] (4) [left of=3] {$p_4$};
          \draw[-latex] (1) to (3);
          \draw[-latex] (2) to (4);
          \draw[-latex] (3) to [out=0, in=0, looseness=1] (2);
          \draw[-latex] (4) to [out=180, in=180, looseness=1] (1);
        \end{tikzpicture} 
        \caption{Deux groupes}
      \end{figure}
    \end{minipage}
  \end{frame}

  \begin{frame}{Objectifs}
    \begin{itemize}[<+- | alert@+>]
      \item Entrée : liste des participants (nom + email + groupe)
      \item Sortie : email
      \item Chiffrement des données
      \item Stockage du tirage
      \item Preuve à divulgation nulle de connaissance
      \item Dénombrement des tirages
    \end{itemize}
  \end{frame}

\section{Tirage au sort}

  \begin{frame}{Cas d'un groupe}
    \begin{figure}
      \begin{tikzpicture}[main/.style = {draw, circle}, scale=1.5, transform shape] 
        \node[main] (1) {$p_1$};
        \node[main] (2) [right of=1] {$p_2$};
        \node[main] (3) [right of=2] {$p_3$};
        \node[main] (4) [right of=3] {$\ldots$};
        \node[main] (5) [right of=4] {$p_n$};
        \draw[-latex] (1) to [out=80, in=100, looseness=1] (2);
        \draw[-latex] (2) to [out=80, in=100, looseness=1] (3);
        \draw[-latex, dashed] (3) to [out=80, in=100, looseness=1] (4);
        \draw[-latex, dashed] (4) to [out=80, in=100, looseness=1] (5);
        \draw[-latex] (5) to [out=270, in=270, looseness=0.5] (1);
      \end{tikzpicture} 
      \caption{Un seul groupe, pas de problème d'existence}
    \end{figure}
  \end{frame}

  \begin{frame}{Cas de plusieurs groupes}
    \begin{figure}
      \begin{tikzpicture}[main/.style = {draw, circle}, scale=1.25, transform shape] 
        \node[main, fill=methodBox!50] (1) {$p_1$};
        \node[main, fill=methodBox!50] (2) [below of=1] {$p_2$};
        \node[main] (3) [right of=2] {$p_3$};
        \node[main] (4) [above of=3] {$p_4$};
        \node[main] (5) [right of=4] {$p_5$};
      \end{tikzpicture} 
      \caption{Cas pathologique}
    \end{figure}
    \pause
    \begin{alertblock}{Condition d'existence du tirage}
      Soit \(G_1,\ldots,G_n\) les différents groupes.

      \[\forall i \in \ie{1, n}, \Card(G_i) \leqslant \sum_{k\in\ie{1, n}\setminus\{i\}}\Card(G_k)\]
    \end{alertblock}
  \end{frame}

  \begin{frame}{Création des paires}
    Soit \(G_1,\ldots,G_n\) les différents groupes.
    \begin{itemize}[<+- | alert@+>]
      \item Mélanger l'ordre des groupes
      \item Pour chaque groupe \(G_i = [p_{i, 1}, \ldots, p_{i, k}]\):
      \begin{enumerate}[<+- | alert@+>]
        \item \(\forall \ell \in \ie{1, k}, p_{i, \ell} \leadsto p_r\) avec \(p_r\hookleftarrow\UU(G_{(i+\ell)\%n})\) et \(\ell \not \equiv 0 [n]\)
        \item Supprimer \(p_r\) de \(G_{(i+\ell)\%n}\)
      \end{enumerate}
    \end{itemize}
  \end{frame}

  \begin{frame}{Exemple}
    \begin{minipage}[h]{0.48\textwidth}
      \begin{figure}
        \begin{tikzpicture}[main/.style = {draw, circle}, scale=0.75, transform shape] 
          \node[main] (1) {$p_1$};
          \node[main] (2) [right of=1] {$p_2$};
          \node[main] (3) [right of=2] {$p_3$};
          \node[main, fill=methodBox!50] (4) [below of=1] {$p_4$};
          \node[main, fill=methodBox!50] (5) [below of=2] {$p_5$};
          \node[main, fill=methodBox!50] (6) [below of=3] {$p_6$};
          \node[main, fill=definitionBox!50] (7) [below of=4] {$p_7$};
          \node[main, fill=definitionBox!50] (8) [below of=5] {$p_8$};
          \node[main, fill=definitionBox!50] (9) [below of=6] {$p_9$};
          \draw[-latex] (1) to (4);
          \draw[-latex] (2) to (7);
          \draw[-latex] (3) to (5);
        \end{tikzpicture} 
      \end{figure}        
    \end{minipage}
    \begin{minipage}[h]{0.48\textwidth}
        \textbf{Étape 1 :} groupe \texttt{[$p_1, p_2, p_3$]}          
        
        $p_1 \leadsto\UU([p_4, p_5, p_6])$
        
        $p_2 \leadsto\UU([p_7, p_8, p_9])$
        
        $p_3 \leadsto\UU([p_5, p_6])$
    \end{minipage}
    \pause
    \bigbreak
    \begin{minipage}[h]{0.48\textwidth}
      \begin{figure}
        \begin{tikzpicture}[main/.style = {draw, circle}, scale=0.75, transform shape] 
          \node[main] (1) {$p_1$};
          \node[main] (2) [right of=1] {$p_2$};
          \node[main] (3) [right of=2] {$p_3$};
          \node[main, fill=pink] (4) [below of=1] {$p_4$};
          \node[main, fill=pink] (5) [below of=2] {$p_5$};
          \node[main, fill=methodBox!50] (6) [below of=3] {$p_6$};
          \node[main, fill=pink] (7) [below of=4] {$p_7$};
          \node[main, fill=definitionBox!50] (8) [below of=5] {$p_8$};
          \node[main, fill=definitionBox!50] (9) [below of=6] {$p_9$};
          \draw[-latex] (4) to (8);
          \draw[-latex] (5) to (1);
          \draw[-latex] (6) to (9);
        \end{tikzpicture} 
      \end{figure}        
    \end{minipage}
    \begin{minipage}[h]{0.48\textwidth}
        \textbf{Étape 2 :} groupe \texttt{[$p_4, p_5, p_6$]}          
        
        $p_4 \leadsto\UU([p_8, p_9])$
        
        $p_5 \leadsto\UU([p_1, p_2, p_3])$
        
        $p_6\leadsto\UU([p_9])$
        
        \color{pink!150}{a déjà reçu un cadeau}
    \end{minipage}
    \pause
    \bigbreak
    \begin{minipage}[h]{0.48\textwidth}
      \begin{figure}
        \begin{tikzpicture}[main/.style = {draw, circle}, scale=0.75, transform shape] 
          \node[main, fill=pink] (1) {$p_1$};
          \node[main] (2) [right of=1] {$p_2$};
          \node[main] (3) [right of=2] {$p_3$};
          \node[main, fill=pink] (4) [below of=1] {$p_4$};
          \node[main, fill=pink] (5) [below of=2] {$p_5$};
          \node[main, fill=methodBox!50] (6) [below of=3] {$p_6$};
          \node[main, fill=pink] (7) [below of=4] {$p_7$};
          \node[main, fill=pink] (8) [below of=5] {$p_8$};
          \node[main, fill=pink] (9) [below of=6] {$p_9$};
          \draw[-latex] (7) to (2);
          \draw[-latex] (8) to (6);
          \draw[-latex] (9) to (3);
        \end{tikzpicture} 
      \end{figure}        
    \end{minipage}
    \begin{minipage}[h]{0.48\textwidth}
        \textbf{Étape 3 :} groupe \texttt{[$p_7, p_8, p_9$]}          
        
        $p_7 \leadsto\UU([p_2, p_3])$
        
        $p_8 \leadsto\UU([p_6])$
        
        $p_9 \leadsto\UU([p_3])$
    \end{minipage}
  \end{frame}

\section{Gestion des données}

  \begin{frame}[fragile]{Représentation des données}
    \begin{table}
      \caption{Format des données d'entrée}
      \begin{tabular}{ccccc}
          \hline
          & Prenom & NOM & Groupe & Email\\
          \hline
          1 & Stanislas & MEZUREUX & MPSI1 & stanmzx@gmail.com \\
          2 & Alice & ESPINOSA & MPSI1 & aliceespinosa29@gmail.com \\
          3 & Bob & \ldots & MPSI2 & \ldots \\
          4 & David & \ldots & MPSI2 & \ldots \\
          \vdots & \vdots & \vdots & \vdots & \vdots \\
          \hline
      \end{tabular}
    \end{table}
    \pause
    \begin{longlisting}
      \caption{Format des données manipulées}
      \begin{minted}[linenos, fontsize=\small]{python}
[[Stanislas, Alice], [Bob, David], ...]

Stanislas = ['Stanislas', 'MEZUREUX', 'MPSI1', 'stanmzx@gmail.com']
      \end{minted}
    \end{longlisting}
  \end{frame}

  \begin{frame}{Définitions}
    \begin{description}[<+->]
      \item[Chiffrer (C)] Transformer un message afin qu'il ne soit lisible qu'à l'aide d'une clé
      \item[Déchiffrer (D)] Opération inverse du chiffrement à l'aide de la clé
      \item[Décrypter (A)] Opération inverse du chiffrement sans la clé
    \end{description}
    \bigbreak
    \begin{figure}
      \begin{tikzpicture}[main/.style = {draw, rectangle}]
        \node[main] (1) {TIPE};
        \node[main] (2) [right of=1] {C};
        \node[main] (3) [right of=2] {\%\$ù£};
        \node[main] (4) [right of=3] {D};
        \node[main] (5) [right of=4] {TIPE};
        \node[main] (6) [below of=3] {A};
        \draw[-latex] (1) to (2);
        \draw[-latex] (2) to (3);
        \draw[-latex] (3) to (4);
        \draw[-latex] (4) to (5);
        \draw[-latex] (6) to (3);
      \end{tikzpicture}
      \caption{Chiffrement symétrique}
    \end{figure}
  \end{frame}

  \begin{frame}{Cryptosystème d'ElGamal}
    \begin{exampleblock}{Cryptosystème asymétique}
      Chiffrement $\leadsto$ clé publique

      Déchiffrement $\leadsto$ clé privée
    \end{exampleblock}
    \pause
    \begin{description}[<+->]
      \item[\texttt{NBITS}] Entier naturel strictement supérieur à 1 
      \item[$q$] Entier premier de \texttt{NBITS} bits
      \item[$g$] Générateur quelconque de \(\Z_{/q\Z}\)
    \end{description}
  \end{frame}

  \begin{frame}{Cryptosystème d'ElGamal}
    \begin{alertblock}{Générer les clés}
      clé privée : \(x\hookleftarrow\UU(\ie{2^{\texttt{NBITS}}, q-1})\)

      clé publique : \((q, g, h)\) avec \(h\egdef g^x\)
    \end{alertblock}
    \pause
    \begin{alertblock}{Chiffer $m\in\N$}
      \(r\hookleftarrow\UU(\ie{2^{\texttt{NBITS}}, q-1})\)

      le message chiffré est \((g^r\text{ mod }q, mh^r\text{ mod }q)\)
    \end{alertblock}
    \pause
    \begin{alertblock}{Déchiffrer $(c_1, c_2)$}
      \(\dfrac{c_2}{c_1^x} = \dfrac{m h^r}{(g^r)^x} = \dfrac{m (g^x)^r}{(g^r)^x}q = \dfrac{m g^{xr}}{g^{xr}} = m\)
    \end{alertblock}
  \end{frame}

  \begin{frame}{Chiffrement du tirage}
    $n$ participants représentés par les entiers de $\ie{1, n}$
    \bigbreak
    $\leadsto$ tirage représenté par une liste de couples (donneur, receveur) puis chiffrée
    \pause
    \bigbreak
    \begin{minipage}[h]{0.48\textwidth}
      \begin{figure}
        \begin{tikzpicture}[main/.style = {draw, circle}] 
          \node[main] (1) {1};
          \node[main] (2) [right of=1] {2};
          \node[main] (3) [below of=2] {3};
          \node[main] (4) [left of=3] {4};
          \draw[-latex] (1) to [out=90, in=90, looseness=1] (2);
          \draw[-latex] (2) to [out=0, in=0, looseness=1] (3);
          \draw[-latex] (3) to [out=270, in=270, looseness=1] (4);
          \draw[-latex] (4) to [out=180, in=180, looseness=1] (1);
        \end{tikzpicture} 
        \caption{Exemple de tirage}
      \end{figure}
    \end{minipage}
    \begin{minipage}[h]{0.48\textwidth}
      $\Rightarrow [(1, 2), (2, 3), (3, 4), (4, 1)]$
    \end{minipage}
    Apprès chiffrement :
    $[((c_1, c_2), (c_3, c_4)), ((c_3, c_4), (c_5, c_6)), ((c_5, c_6), (c_7, c_8)), ((c_7, c_8), (c_1, c_2))]$
  \end{frame}

  \begin{frame}{Preuve à divulgation nulle de connaissance}
    Validité du tirage : chaque participant donne et reçoit exactement 1 cadeau
    \begin{alertblock}{Condition de validité du tirage}
      Soit $[(d_1, r_1), \ldots, (d_n, r_n)]$ le tirage.
        \(\text{le tirage est valide} \iff
        \begin{cases}
            \displaystyle\prod_{i=1}^nd_i=n ! \\
            \displaystyle\prod_{i=1}^nr_i=n !
        \end{cases}\)
    \end{alertblock}
  \end{frame}

  \begin{frame}{Chiffrement homomorphe}
    \begin{exampleblock}{Définition}
      Le cryptographe peut effectuer des opérations entre entiers chiffrés.

      \textsc{ElGamal} est homomorphe vis-à-vis de la multiplication.
    \end{exampleblock}
    Soit $m_1$ et $m_2$ chiffrés respectivement par $(c_{1, 1}, c_{1, 2})$ et $(c_{2, 1}, c_{2, 2})$.
    \begin{align*}
      \dfrac{c_{1, 2} c_{2, 2}}{(c_{1, 1} c_{2, 1})^x}
      &= \dfrac{m_1 h^{r_1} m_2  h^{r_2}}{(g^{r_1} g^{r_2})^x} \\
      &= \dfrac{m_1m_2 h^{r_1+r_2}}{(g^{r_1+r_2})^x} \\
      &= \dfrac{m_1m_2(g^x)^{r_1+r_2}}{(g^{r_1+r_2})^x} \\
      &= m_1m_2\text{ mod }q
    \end{align*}
  \end{frame}

\section{Dénombrement des tirages}

  \begin{frame}{Estimation algorithmique}
    $p$ : nombre de personnes par groupe 
    
    $n$ : nombre de groupes
    \begin{figure}
      \begin{tikzpicture}
        \begin{semilogyaxis}[
          mlineplot,
          width=0.9\textwidth,
          xlabel=$p$, xlabel style={at=(current axis.right of origin), anchor=west},
          height=6cm,
        ]

        \addplot [
          domain=0:10, 
          samples=11, 
          color=TolColor2,
        ]
        {x!};
        \addlegendentry{1 groupe}
        \addplot [
          domain=0:10, 
          samples=11, 
          color=TolColor7,
        ]
        {(x!)^2};
        \addlegendentry{2 groupes}
        \addplot [
          domain=0:10, 
          samples=11, 
          color=TolColor9,
        ]
        {(x!)^3};
        \addlegendentry{3 groupes}

        \end{semilogyaxis}
      \end{tikzpicture}
      \caption{Nombre de tirages possibles}
    \end{figure}
  \end{frame}

\section{Conclusion}

  \begin{frame}{Conclusion}
    \begin{itemize}[<+- | alert@+>]
      \item Algorithme opérationnel
      \item Prévention contre la triche
      \item Nombreux domaines d'application
    \end{itemize}
  \end{frame}

\plain{Questions?}

\section{Annexes}

  \begin{frame}[fragile, noframenumbering]{Code chiffrement homomorphe}
    \inputminted[linenos, breaklines, autogobble, fontsize=\small, firstline=1, lastline=16]{python}{../elgamal.py}
  \end{frame}

  \begin{frame}[fragile, noframenumbering]{Code chiffrement homomorphe}
    \inputminted[linenos, breaklines, autogobble, fontsize=\small, firstline=17, lastline=31]{python}{../elgamal.py}
  \end{frame}

  \begin{frame}[fragile, noframenumbering]{Code chiffrement homomorphe}
    \inputminted[linenos, breaklines, autogobble, fontsize=\small, firstline=32, lastline=47]{python}{../elgamal.py}
  \end{frame}

  \begin{frame}[fragile, noframenumbering]{Code chiffrement homomorphe}
    \inputminted[linenos, breaklines, autogobble, fontsize=\small, firstline=48, lastline=63]{python}{../elgamal.py}
  \end{frame}

  \begin{frame}[fragile, noframenumbering]{Code chiffrement homomorphe}
    \inputminted[linenos, breaklines, autogobble, fontsize=\small, firstline=64, lastline=79]{python}{../elgamal.py}
  \end{frame}

  \begin{frame}[fragile, noframenumbering]{Code chiffrement homomorphe}
    \inputminted[linenos, breaklines, autogobble, fontsize=\small, firstline=80, lastline=88]{python}{../elgamal.py}
  \end{frame}

  \begin{frame}[fragile, noframenumbering]{Code Secret Santa}
    \inputminted[linenos, breaklines, autogobble, fontsize=\small, firstline=1, lastline=16]{python}{../SecretSanta.py}
  \end{frame}

  \begin{frame}[fragile, noframenumbering]{Code Secret Santa}
    \inputminted[linenos, breaklines, autogobble, fontsize=\small, firstline=17, lastline=31]{python}{../SecretSanta.py}
  \end{frame}

  \begin{frame}[fragile, noframenumbering]{Code Secret Santa}
    \inputminted[linenos, breaklines, autogobble, fontsize=\small, firstline=32, lastline=47]{python}{../SecretSanta.py}
  \end{frame}

  \begin{frame}[fragile, noframenumbering]{Code Secret Santa}
    \inputminted[linenos, breaklines, autogobble, fontsize=\small, firstline=48, lastline=63]{python}{../SecretSanta.py}
  \end{frame}

  \begin{frame}[fragile, noframenumbering]{Code Secret Santa}
    \inputminted[linenos, breaklines, autogobble, fontsize=\small, firstline=64, lastline=79]{python}{../SecretSanta.py}
  \end{frame}

  \begin{frame}[fragile, noframenumbering]{Code Secret Santa}
    \inputminted[linenos, breaklines, autogobble, fontsize=\small, firstline=80, lastline=94]{python}{../SecretSanta.py}
  \end{frame}

  \begin{frame}[fragile, noframenumbering]{Code Secret Santa}
    \inputminted[linenos, breaklines, autogobble, fontsize=\small, firstline=95, lastline=107]{python}{../SecretSanta.py}
  \end{frame}

  \begin{frame}[fragile, noframenumbering]{Code Secret Santa}
    \inputminted[linenos, breaklines, autogobble, fontsize=\small, firstline=108, lastline=118]{python}{../SecretSanta.py}
  \end{frame}

  \begin{frame}[fragile, noframenumbering]{Code Secret Santa}
    \inputminted[linenos, breaklines, autogobble, fontsize=\small, firstline=119, lastline=133]{python}{../SecretSanta.py}
  \end{frame}

  \begin{frame}[fragile, noframenumbering]{Code Secret Santa}
    \inputminted[linenos, breaklines, autogobble, fontsize=\small, firstline=134, lastline=148]{python}{../SecretSanta.py}
  \end{frame}

  \begin{frame}[fragile, noframenumbering]{Code Secret Santa}
    \inputminted[linenos, breaklines, autogobble, fontsize=\small, firstline=149, lastline=161]{python}{../SecretSanta.py}
  \end{frame}

  \begin{frame}[fragile, noframenumbering]{Code dénombrement tirages}
    \inputminted[linenos, breaklines, autogobble, fontsize=\small, firstline=1, lastline=16]{python}{../draw_counter.py}
  \end{frame}

  \begin{frame}[fragile, noframenumbering]{Code dénombrement tirages}
    \inputminted[linenos, breaklines, autogobble, fontsize=\small, firstline=17, lastline=32]{python}{../draw_counter.py}
  \end{frame}

  \begin{frame}[fragile, noframenumbering]{Code dénombrement tirages}
    \inputminted[linenos, breaklines, autogobble, fontsize=\small, firstline=33, lastline=47]{python}{../draw_counter.py}
  \end{frame}

\end{document}
